%
% Copyright 2014, General Dynamics C4 Systems
%
% This software may be distributed and modified according to the terms of
% the GNU General Public License version 2. Note that NO WARRANTY is provided.
% See "LICENSE_GPLv2.txt" for details.
%
% @TAG(GD_GPL)
%

%macros for API documentation
\newcommand{\param}[3]{\texttt{#1}&\texttt{#2}&#3\\ }

\newcommand{\inputapidoc}[1] {\input{parts/api/#1.tex}}

\newcommand{\apidoc}[7]
{
    \subsection{\label{api:#1}#2}

    \texttt{#4}
    \vspace*{6pt}

    #3

    \begin{center}
    \begin{minipage}{0.95\textwidth}
    \begin{tabularx}{\textwidth}{llX}
    \toprule
    \textbf{Type} & \textbf{Name} & \textbf{Description} \\
    \midrule
    #5
    \bottomrule
    \end{tabularx}
    \end{minipage}
    \end{center}

    \textit{Return value:} #6 \par

    \textit{Description:} #7 \par

    \vfill
}

%Common parameter descriptions
\newcommand{\destcspacedesc}{CPTR to the CNode that forms the root of the destination CSpace. Must be at a depth of 32.}
\newcommand{\destindexdesc}{CPTR to the destination slot. Resolved from the root of the destination CSpace.}
\newcommand{\destdepthdesc}{Number of bits of dest\_index to resolve to find the destination slot.}
\newcommand{\srccspacedesc}{CPTR to the CNode that forms the root of the source CSpace. Must be at a depth of 32.}
\newcommand{\srcindexdesc}{CPTR to the source slot. Resolved from the root of the source CSpace.}
\newcommand{\srcdepthdesc}{Number of bits of src\_index to resolve to find the source slot.}
\newcommand{\rightsdesc}{The rights inherited by the new capability. Possible values for this type are given in \autoref{sec:cap_rights}.}
\newcommand{\badgedesc}{Badge to be applied to the new capability.}

\newcommand{\cspacedesc}{CPTR to the CNode at the root of the CSpace where the capability will be found. Must be at a depth of 32.}
\newcommand{\indexdesc}{CPTR to the capability. Resolved from the root of the \_service parameter.}
\newcommand{\depthdesc}{Number of bits of index to resolve to find the capability being operated on.}

\newcommand{\ioportcapdesc}{An IO port capability.}
\newcommand{\ioportdescread}{The port to read from.}
\newcommand{\ioportdescwrite}{The port to write to.}
\newcommand{\ioportdatadesc}{Data to write to the IO port.}

\newcommand{\pagecapdesc}{Capability to the page to map.}
\newcommand{\pdcapdesc}{Capability to the VSpace which will contain the mapping.}
\newcommand{\vaddrdesc}{Virtual address to map the page into.}
\newcommand{\vmcaprightsdesc}{Rights for the mapping. Possible values for this type are given in \autoref{sec:cap_rights}.}
\newcommand{\vmattribsdescarm}{VM Attributes for the mapping. Possible values for this type are given in \autoref{ch:vspace}.}

\newcommand{\debugargbpnumshortdesc}{
The API-ID of a target breakpoint. This ID will be a positive integer, with
values ranging from \texttt{0} to \texttt{seL4\_NumHWBreakpoints - 1}.
}

\newcommand{\debugargvaddrshortdesc}{A virtual address which forms part of the
match conditions for the triggering of the breakpoint.
}

\newcommand{\debugargtypeshortdesc}{One of: \texttt{seL4\_InstructionBreakpoint}, which specifies
that the breakpoint should occur on instruction execution at the specified
\texttt{vaddr} or \texttt{seL4\_DataBreakpoint}, which states that the breakpoint
should occur on data access at the specified \texttt{vaddr}.
}

\newcommand{\debugargsizeshortdesc}{A positive integer indicating the
trigger-span of the watchpoint. Must be zero when 'type' is \texttt{seL4\_InstructionBreakpoint}.
}

\newcommand{\debugargrwshortdesc}{One of \texttt{seL4\_BreakOnRead}, meaning the breakpoint will only be
triggered on read-access; \texttt{seL4\_BreakOnWrite} meaning the
breakpoint will only be triggered on write-access, and
\texttt{seL4\_BreakOnReadWrite} meaning the breakpoint will be triggered on
any access.
}

\ifxeightsix
\newcommand{\vmattribsdescintel}{VM Attributes for the
  mapping. Possible values for this type are given in \autoref{ch:vspace}. }
\fi

\newcommand{\tcbcapdesc}{Capability to the TCB which is being operated on.}
\newcommand{\threadschedcontextcapdesc}{Capability to the scheduling context that the TCB should run on. If the scheduling context is already bound to a notification or TCB that is not this TCB this operation will fail. Similarly, if this TCB is already bound to a scheduling context that is not this scheduling context, this will also fail. }
\newcommand{\schedcontrolcapdesc}{The SchedControl capability.}
\newcommand{\schedcontextcapdesc}{Capability to the scheduling context which is being operated on.}
\newcommand{\budgetdesc}{Timeslice in microseconds, when the budget expires the thread will be preempted.}
\newcommand{\perioddesc}{Period in microseconds, the budget is replenished every time the period exipires.}
\newcommand{\datadesc}{A word of data to identify the owner of the scheduling context that a thread faulted on to a temporal fault handler.}

\newcommand{\irqhandlercapdesc}{The IRQ handler capability.}
\newcommand{\invokedcapdesc}{The capability to be invoked.}
\newcommand{\messageinfodesc}{The messageinfo structure for the IPC.}
\newcommand{\senderdesc}{The address to write sender information to. The sender information is the badge of the endpoint capability that was invoked by the sender, or the notification word of the notification object that was signalled. This parameter is ignored if \texttt{NULL}.}
\newcommand{\resumetargetdesc}{The invocation should also resume the destination thread.}
\newcommand{\suspendsourcedesc}{The invocation should also suspend the source thread.}
\newcommand{\archflagsdesc}{Architecture dependent flags. These have no meaning on \ifxeightsix{either IA-32 or}\fi{} ARM.}
\newcommand{\criticalitydesc}{The system criticality level. Threads with criticalities below the system criticality level will not be chosen by the scheduler.}
\newcommand{\threadpriodesc}{The thread's new priority.}
\newcommand{\threadmaxpriodesc}{The thread's new maximum controlled priority.}
\newcommand{\threadcriticalitydesc}{The thread's new criticality.}
\newcommand{\threadmccdesc}{The thread's new maximum controlled criticality.}
\newcommand{\threadcombinedpriodesc}{A \texttt{seL4\_Prio\_t} structure as described in \autoref{sec:sched}.}
\newcommand{\threadcspacerootdesc}{The new CSpace root.}
\newcommand{\threadcspacedatadesc}{Optionally set the guard and guard size of the new root CNode. If set to zero, this parameter has no effect.}
\newcommand{\threadvspacerootdesc}{The new VSpace root.}
\newcommand{\threadvspacedatadesc}{Has no effect on \ifxeightsix{IA-32 or}\fi{} ARM processors.}
\newcommand{\threadbufferdesc}{Location of the thread's IPC buffer. Must be 512-byte aligned. The IPC buffer may not cross a page boundary.}
\newcommand{\threadbufferpagedesc}{Capability to a page containing the thread's IPC buffer.}
\newcommand{\excepthanddesc}{CPTR to the endpoint which receives IPCs when this thread faults. This capability is in the caller's cspace.}
\newcommand{\temporalexcepthanddesc}{CPTR to the endpoint which receives IPCs when this thread triggers a temporal fault. This capability is in the caller's cspace.}

\newcommand{\asidassignpooldesc}{The ASID pool which is being assigned to. Must not be full. Each ASID pool can contain 1024 entries.}
\newcommand{\asidassignpddesc}{The page directory that is being assigned to an ASID pool. Must not already be assigned to an ASID pool.}

%Return value descriptions
\newcommand{\messageinforetdesc}{A \texttt{seL4\_MessageInfo\_t} structure as described in \autoref{sec:messageinfo}.}
\newcommand{\noret}{This method does not return anything.}
\newcommand{\errorenumdesc}{A return value of \texttt{0} indicates success. A non-zero value indicates that an error occurred. See \autoref{sec:errors} for a description of the message register and tag contents upon error.}
\newcommand{\consumeddesc}{\texttt{consumed}, which is the amount of time that threads have been executing using this scheduling context since \apifunc{seL4\_SchedContext\_YieldTo}{schedcontext_yieldto} or \apifunc{seL4\_SchedContext\_Consumed}{schedcontext_consumed} was called, and \texttt{error}, an error code, see \autoref{sec:errors}.}
\newcommand{\pagegetaddresstdesc}{struct that contains \texttt{seL4\_Word paddr}, which holds the physical address of the page, and \texttt{int error}. See \autoref{sec:errors} for a description of the message register and tag contents upon error.}
\newcommand{\tcbgetbreakpointtdesc}{Struct that contains
'\texttt{seL4\_Error error}', an seL4 API error value,
'\texttt{seL4\_Word vaddr}', the virtual address at which the breakpoint will currently
be triggered;
'\texttt{seL4\_Word type}', the type of operation which will currently trigger the
breakpoint, whether instruction execution, or data access;
'\texttt{seL4\_Word size}', integer value for the span-size of the breakpoint.
Usually a multiple of two (1, 2, 4, etc.);
'\texttt{seL4\_Word rw}', the access direction that will currently trigger the breakpoint,
whether read, write, or both and
'\texttt{seL4\_Bool is\_enabled}', which indicates whether or not the breakpoint
will currently be triggered if the match conditions are met.
}
\newcommand{\tcbconfiguresinglesteppingtdesc}{Struct that contains
'\texttt{seL4\_Error error}', an seL4 API error value,
'\texttt{seL4\_Bool bp\_was\_consumed}', a boolean which indicates whether or not the \texttt{bp\_num}
breakpoint ID that was passed to the function, was consumed in the setup of the single-stepping
functionality: if this is \texttt{true}, the caller should not attempt to re-use \texttt{bp\_num}
until it has disabled the single-stepping functionality via a subsequent call to
seL4\_TCB\_ConfigureSingleStepping with an \texttt{nun\_instructions} argument of 0.
}

\section{Error Codes}
\label{sec:errors}

Invoking a capability with invalid parameters will result in an error.
seL4 system calls return an error code in the message tag and a short
error description in the message registers to aid the programmer in
determining the cause of errors.\\

\subsection{Invalid Argument}

A non-capability argument is invalid.

\begin{tabularx}{\textwidth}{p{0.25\textwidth}X}
\toprule
    Field & Meaning \\
\midrule
    \ipcbloc{Label} & \enummem{seL4\_InvalidArgument} \\
    \ipcbloc{IPCBuffer[0]} & Invalid argument number \\
\bottomrule
\end{tabularx}
\vfill

\subsection{Invalid Capability}

A capability argument is invalid.

\begin{tabularx}{\textwidth}{p{0.25\textwidth}X}
\toprule
    Field & Meaning \\
\midrule
    \ipcbloc{Label} & \enummem{seL4\_InvalidCapability} \\
    \ipcbloc{IPCBuffer[0]} & Invalid capability argument number \\
\bottomrule
\end{tabularx}
\vfill

\subsection{Illegal Operation}

The requested operation is not permitted.

\begin{tabularx}{\textwidth}{p{0.25\textwidth}X}
\toprule
    Field & Meaning \\
\midrule
    \ipcbloc{Label} & \enummem{seL4\_IllegalOperation} \\
\bottomrule
\end{tabularx}
\vfill

\subsection{Range Error}

An argument is out of the allowed range.

\begin{tabularx}{\textwidth}{p{0.25\textwidth}X}
\toprule
    Field & Meaning \\
\midrule
    \ipcbloc{Label} & \enummem{seL4\_RangeError} \\
    \ipcbloc{IPCBuffer[0]} & Minimum allowed value \\
    \ipcbloc{IPCBuffer[1]} & Maximum allowed value \\
\bottomrule
\end{tabularx}
\vfill

\subsection{Alignment Error}

A supplied argument does not meet the alignment requirements.

\begin{tabularx}{\textwidth}{p{0.25\textwidth}X}
\toprule
    Field & Meaning \\
\midrule
    \ipcbloc{Label} & \enummem{seL4\_AlignmentError} \\
\bottomrule
\end{tabularx}
\vfill

\subsection{Failed Lookup}

A capability could not be looked up.

\begin{tabularx}{\textwidth}{p{0.25\textwidth}X}
\toprule
    Field & Meaning \\
\midrule
    \ipcbloc{Label} & \enummem{seL4\_FailedLookup} \\
    \ipcbloc{IPCBuffer[0]} & 1 if the lookup failed for a source capability, 0 otherwise\\
    \ipcbloc{IPCBuffer[1]} & Type of lookup failure\\
    \ipcbloc{IPCBuffer[2..]} & Lookup failure description as described in \autoref{sec:lookup_fail_desc}\\
\bottomrule
\end{tabularx}
\vfill

\subsection{Delete First}

A destination slot specified in the syscall arguments is occupied.

\begin{tabularx}{\textwidth}{p{0.25\textwidth}X}
\toprule
    Field & Meaning \\
\midrule
    \ipcbloc{Label} & \enummem{seL4\_DeleteFirst} \\
\bottomrule
\end{tabularx}
\vfill

\subsection{Revoke First}

The object currently has other objects derived from it and the requested
invocation cannot be performed until either these objects are deleted or
the revoke invocation is performed on the capability.

\begin{tabularx}{\textwidth}{p{0.25\textwidth}X}
\toprule
    Field & Meaning \\
\midrule
    \ipcbloc{Label} & \enummem{seL4\_RevokeFirst} \\
\bottomrule
\end{tabularx}
\vfill

\subsection{Not Enough Memory}

The \obj{Untyped Memory} object does not have enough unallocated space to
complete the \apifunc{seL4\_Untyped\_Retype}{untyped_retype} request.

\begin{tabularx}{\textwidth}{p{0.25\textwidth}X}
\toprule
    Field & Meaning \\
\midrule
    \ipcbloc{Label} & \enummem{seL4\_NotEnoughMemory} \\
    \ipcbloc{IPCBuffer[0]} & Amount of memory available in bytes\\
\bottomrule
\end{tabularx}
\vfill

\section{System Calls}
\inputapidoc{sel4_send}
\inputapidoc{sel4_recv}
\inputapidoc{sel4_call}
\inputapidoc{sel4_reply}
\inputapidoc{sel4_nbsend}
\inputapidoc{sel4_replyrecv}
\inputapidoc{sel4_nbrecv}
\inputapidoc{sel4_signal}
\inputapidoc{sel4_wait}
\inputapidoc{sel4_poll}
\inputapidoc{sel4_signalrecv}
\clearpage

\section{Architecture-Independent Object Methods}
\label{sec:kobj_api}
\inputapidoc{cnode_copy}
\inputapidoc{cnode_delete}
\inputapidoc{cnode_mint}
\inputapidoc{cnode_move}
\inputapidoc{cnode_mutate}
\inputapidoc{cnode_recycle}
\inputapidoc{cnode_revoke}
\inputapidoc{cnode_rotate}
\inputapidoc{cnode_swapcaller}
\inputapidoc{cnode_swaptcbcaller}
\inputapidoc{debug_halt}
\inputapidoc{debug_putchar}
\inputapidoc{irq_controlget}
\inputapidoc{irq_handleracknowledge}
\inputapidoc{irq_handlerclear}
\inputapidoc{irq_handlersetnotification}
\inputapidoc{tcb_bindnotification}
\inputapidoc{tcb_unbindnotification}
\inputapidoc{tcb_configure}
\inputapidoc{tcb_copyregisters}
\inputapidoc{tcb_readregisters}
\inputapidoc{tcb_resume}
\inputapidoc{tcb_setipcbuffer}
\inputapidoc{tcb_setpriority}
\inputapidoc{tcb_setmcpriority}
\inputapidoc{tcb_setcriticality}
\inputapidoc{tcb_setmcc}
\inputapidoc{tcb_setspace}
\inputapidoc{tcb_suspend}
\inputapidoc{tcb_writeregisters}
\inputapidoc{tcb_setbreakpoint}
\inputapidoc{tcb_getbreakpoint}
\inputapidoc{tcb_unsetbreakpoint}
\inputapidoc{tcb_configuresinglestepping}
\inputapidoc{untyped_retype}
\inputapidoc{schedcontrol_configure}
\inputapidoc{schedcontrol_setcriticality}
\inputapidoc{schedcontext_yield}
\inputapidoc{schedcontext_yieldto}
\inputapidoc{schedcontext_consumed}
\inputapidoc{schedcontext_bind}
\inputapidoc{schedcontext_unbindobject}
\inputapidoc{schedcontext_unbind}
\ifxeightsix
\clearpage

\section{IA-32-Specific Object Methods}
\label{sec:kobj_api_intel}
\inputapidoc{ia32_ASID_controlmakepool}
\inputapidoc{ia32_ASID_poolassign}
\inputapidoc{ia32_IO_portin8}
\inputapidoc{ia32_IO_portin16}
\inputapidoc{ia32_IO_portin32}
\inputapidoc{ia32_IO_portout8}
\inputapidoc{ia32_IO_portout16}
\inputapidoc{ia32_IO_portout32}
\inputapidoc{ia32_io_pagetable_map}
\inputapidoc{ia32_page_mapio}
\inputapidoc{ia32_page_map}
\inputapidoc{ia32_page_remap}
\inputapidoc{ia32_page_unmap}
\inputapidoc{ia32_page_getaddress}
\inputapidoc{ia32_pagetable_map}
\inputapidoc{ia32_pagetable_unmap}
\inputapidoc{ia32_pagedirectory_getstatusbits}
\fi

\clearpage

\section{ARM-Specific Object Methods}
\label{sec:kobj_api_arm}
\inputapidoc{arm_asidcontrol_makepool}
\inputapidoc{arm_asidpool_assign}
\inputapidoc{arm_page_cleandata}
\inputapidoc{arm_page_invalidatedata}
\inputapidoc{arm_page_cleaninvalidatedata}
\inputapidoc{arm_page_unifyinstruction}
\inputapidoc{arm_page_map}
\inputapidoc{arm_page_remap}
\inputapidoc{arm_page_unmap}
\inputapidoc{arm_page_getaddress}
\inputapidoc{arm_pagetable_map}
\inputapidoc{arm_pagetable_unmap}
