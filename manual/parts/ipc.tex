%
% Copyright 2014, General Dynamics C4 Systems
%
% This software may be distributed and modified according to the terms of
% the GNU General Public License version 2. Note that NO WARRANTY is provided.
% See "LICENSE_GPLv2.txt" for details.
%
% @TAG(GD_GPL)
%

\chapter{\label{ch:ipc}Inter-process Communication}

The seL4 microkernel provides a message-passing mechanism for communication
between threads. The mechanism is also used for communication with
kernel-provided services. Messages are sent by invoking a capability to a
kernel object. Messages sent to \emph{synchronous} (\obj{Endpoint}) and
\emph{asynchronous} (\obj{Async\-EP}) IPC endpoints are destined for other
threads; messages sent to objects of other types are processed by the kernel. This
chapter describes the common message format, the two types of endpoints,
and how they can be used for communication between applications.

\section{Message Registers}
\label{sec:messageinfo}

Each message contains a number of message words and optionally a number of
capabilities.
The message words are sent to or received from a thread by placing them in its \emph{message registers}.
The message registers are numbered and the first few message registers are implemented
using physical CPU registers, while the rest are backed by a fixed region of
memory called the \emph{IPC buffer}.
The reason for this design is efficiency:
very short messages need not use the memory.
The physical CPU registers used for the
message registers are described in \ifxeightsix\autoref{tbl:mrs_x86} for x86
% x86 is used rather than IA-32 to be vendor neutral.
and \fi \autoref{tbl:mrs_arm} for ARM.
The IPC buffer is assigned to the calling thread (see \autoref{sec:threads} and \autoref{api:tcb_setipcbuffer}).
%FIXME: seL4_TCB_SetIPCBuffer is only mentioned in the API reference!

\ifxeightsix
\begin{table}[htb]
    \begin{center}
        \begin{tabular}{lc}
            \toprule
            Role                                    & CPU Register \\
            \midrule
            Capability register \emph{(in)}         & \texttt{ebx} \\
            Badge register \emph{(out)}             & \texttt{ebx} \\
            Message tag \emph{(in/out)}     & \texttt{esi} \\
            Message register 1 \emph{(in/out)}      & \texttt{edi} \\
            Message register 2 \emph{(in/out)}      & \texttt{ebp} \\
            \bottomrule
        \end{tabular}
    \caption{\label{tbl:mrs_x86}Physical register allocation for IPC messages
    on the x86 architecture.}
    \end{center}
\end{table}
\fi

\begin{table}[htb]
    \begin{center}
        \begin{tabular}{lc}
            \toprule
            Role                                    & CPU Register \\
            \midrule
            Capability register \emph{(in)}         & \texttt{r0} \\
            Badge register \emph{(out)}             & \texttt{r0} \\
            Message tag \emph{(in/out)}     & \texttt{r1} \\
            Message register 1--4 \emph{(in/out)}   & \texttt{r2} -- \texttt{r5} \\
            \bottomrule
        \end{tabular}
    \caption{\label{tbl:mrs_arm}Physical register allocation for IPC messages on the ARM architecture.}
    \end{center}
\end{table}

Every IPC message also has a tag (structure \texttt{seL4\_MessageInfo\_t}).  The
tag consists of four fields: the label, message length, number of capabilities
(the \texttt{extraCaps} field) and the \texttt{capsUnwrapped} field.  The
message length and number of capabilities determine either the number of
message registers and capabilities that the sending thread wishes to transfer,
or the number of message registers and capabilities that were actually
transferred. The label is not interpreted by the
kernel and is passed unmodified as the first data payload of the message. The
label may, for example, be used to specify a requested operation. The
\texttt{capsUnwrapped} field is used only on the receive side, to indicate the
manner in which capabilities were received. It is described in
\autoref{sec:cap-transfer}.

% FIXME: a little too low-level, perhaps?

\newcommand{\ipcparam}[4]{\texttt{#1} \emph{#2}&\texttt{#3}&#4\\ }
\begin{table}[htb]
    \begin{center}
    \begin{tabularx}{\textwidth}{p{0.28\textwidth}p{0.18\textwidth}X}
      \toprule
      \textbf{Type} & \textbf{Name} & \textbf{Description} \\
      \midrule
        \ipcparam{seL4\_MessageInfo\_t}{}{tag}{Message tag}
        \ipcparam{seL4\_Word[]}{}{msg}{Message contents}
        \ipcparam{seL4\_Word}{}{userData}{Base address of the structure, used by
        supporting user libraries}
        \ipcparam{seL4\_CPtr[]}{(in)}{caps}{Capabilities to transfer}
        \ipcparam{seL4\_CapData\_t[]}{(out)}{badges}{Badges for
        endpoint capabilities received}
        \ipcparam{seL4\_CPtr}{}{receiveCNode}{CPTR to a CNode from which to
        find
        the receive slot}
        \ipcparam{seL4\_CPtr}{}{receiveIndex}{CPTR to the receive slot
        relative to \texttt{receiveCNode}}
        \ipcparam{seL4\_Word}{}{receiveDepth}{Number of bits of
        \texttt{receiveIndex} to
        use}
        \bottomrule
      \end{tabularx}
    \caption{\label{tbl:ipcbuffer}Fields of the
      \texttt{seL4\_IPCBuffer} structure.  Note that
      \texttt{badges} and \texttt{caps} use the same area of memory in
      the structure.}
    \end{center}
\end{table}

The kernel assumes that the IPC buffer contains a structure of type
\texttt{seL4\_IPCBuffer} as defined in \autoref{tbl:ipcbuffer}. The
kernel uses as many physical registers as possible to transfer IPC
messages. When more arguments are transferred than physical message
registers are available, the kernel begins using the IPC buffer's
\texttt{msg} field to transfer arguments. However, it leaves room in
this array for the physical message registers. For example, if an IPC
transfer or kernel object invocation required
4 message registers (and there are only 2 physical message registers
available on this architecture) then arguments 1 and 2 would be
transferred via message registers and arguments 3 and 4 would be in
\texttt{msg[2]} and \texttt{msg[3]}.
This allows the user-level object-invocation stubs to copy the arguments passed in physical registers to
the space left in the \texttt{msg} array if desired.
The situation is similar for the tag field.
There is space for this field in the \texttt{seL4\_IPCBuffer} structure, which the kernel ignores.
User level stubs
may wish to copy the message tag from its CPU register to this field, although
the user level stubs provided with the kernel do not do this.

\section{Synchronous Endpoints}

Synchronous endpoints (or simply \obj{Endpoint}s) allow a small amount
of data and small number of capabilities (namely the IPC buffer) to be transferred between two
threads. \obj{Endpoints} are called
`synchronous' because a message transfer will not take
place until both the sender is ready to send the message and a receiver is
ready to receive it.

\obj{Endpoint} objects are invoked directly using the seL4 system calls
described in \autoref{sec:syscalls}. Note that \obj{Endpoint} objects may queue
threads either to send or to receive. If no receiver is ready, threads
performing the \apifunc{seL4\_Send}{sel4_send} or \apifunc{seL4\_Call}{sel4_call}
system calls will wait in a queue for the first available receiver. Likewise if
no sender is ready, threads performing the \apifunc{seL4\_Wait}{sel4_wait}
system call or the second half of \apifunc{seL4\_ReplyWait}{sel4_replywait}
will wait for the first available sender.

\subsection{Endpoint Badges}
\label{sec:ep-badges}

Synchronous endpoint capabilities may be \emph{minted} to
create a new endpoint capability with a \emph{badge} attached to it. The
badge is a word of data that is associated with a particular endpoint
capability. When a message is sent to an endpoint using a badged
capability, the badge is transferred to the receiving thread's
\texttt{badge} register.

An endpoint capability with a zero badge is said to be \emph{unbadged}.
Such a capability can be badged with the \apifunc{seL4\_CNode\_Mutate}{cnode_mutate} or \apifunc{seL4\_CNode\_Mint}{cnode_mint}
invocations on the \obj{CNode} containing the capability. Endpoint
capabilities with badges cannot be unbadged, rebadged or used to create
child capabilities with different badges.

\subsection{Capability Transfer}
\label{sec:cap-transfer}

Messages in seL4 may contain capabilities. Messages containing capabilities can
be sent across synchronous endpoints, provided that the endpoint capability
invoked by the sending thread has Grant rights. An attempt to transfer
capabilities without Grant rights can still result in the message being sent,
but no capabilities will be transferred.

Capabilities to be sent with a message are specified in the sending thread's
IPC buffer in the \texttt{caps} field. Each entry in that array is interpreted
as a CPTR in the sending thread's capability space. The number of capabilities
to send is specified in the \texttt{extraCaps} field of the message tag.

The receiver specifies the slot,
in which it is willing to receive a capability, with three fields within the IPC buffer: \texttt{receiveCNode}, \texttt{receiveIndex} and \texttt{receiveDepth}.
These fields specify the root CNode, capability address and number of bits to resolve, respectively, to find
the slot in which to put the capability. Capability
addressing is described in \autoref{sec:cap_addressing}.

A received capability has the same rights as the original except if the receiving endpoint capability does not have the Write right.
In this case, the rights on the sent capability are \emph{diminished}, by
removing from the received copy of that capability the Write right.

Note that receiving threads may specify only one receive slot, whereas a
sending thread may include multiple capabilities in the message. Messages
containing more than one capability may be interpreted by kernel objects. They
may also be sent to receiving threads in the case where some of the extra
capabilities in the message can be \emph{unwrapped}.

If the n-th capability in the message refers to the same endpoint as the one
the
message is being sent through, it is \emph{unwrapped}. Its badge is placed in
the n-th
position of the receiver's badges array and the n-th bit (counting from the
least significant) is set in the \texttt{capsUnwrapped} field of the message
tag. The capability itself is not transferred, so the receive slot may be used
for another one.

If a receiver gets a message whose tag has an \texttt{extraCaps} of 2 and a
\texttt{capsUnwrapped} of 2, then the first capability in the message was
transferred to the specified receive slot and the second capability was
unwrapped, placing its badge in \texttt{badges[1]}. There may have been a
third capability in the sender's message which could not be unwrapped.

\subsection{Errors}

% FIXME : will this actually issue a fault? TS: in the send phase? yes, yes it will

Errors in capability transfers can occur at two places: in the send
phase or in the receive phase. In the send phase, all capabilities that
the caller is attempting to send are looked up to ensure that they exist
before the send is initiated in the kernel. If the lookup fails for any
reason, \apifunc{seL4\_Send}{sel4_send} and \apifunc{seL4\_Call}{sel4_call} system calls immediately abort and
no IPC or capability transfer takes place. The system call will return
a lookup failure error as described in \autoref{sec:errors}.

In the receive phase, seL4 transfers capabilities in the order that they
are found in the sending thread's IPC buffer \texttt{caps} array
and terminates as soon as an error is encountered. Possible error
conditions are:

\begin{itemize}
    \item A source capability cannot be looked up. Although the presence
    of the source capabilities is checked when the sending thread
    performs the send system call, this error may still occur. The sending
    thread may have been blocked on the endpoint for some time before it
    was paired with a receiving thread. During this time, its
    CSpace may have changed and the source capability pointers may
    no longer be valid.

    \item The destination slot cannot be looked up. Unlike the send
    system call, the \apifunc{seL4\_Wait}{sel4_wait} system call does not check that the
    destination slot exists and is empty before it initiates the wait.
    Hence, the \apifunc{seL4\_Wait}{sel4_wait} system call will not fail with an error if the
    destination slot is invalid and will instead transfer badged
    capabilities until an attempt to save a capability to the
    destination slot is made.

    \item The capability being transferred cannot be derived. See
    \autoref{sec:cap_derivation} for details.
\end{itemize}

An error will not void the entire transfer, it will just end it
prematurely. The capabilities processed before the failure are still
transferred and the \texttt{extraCaps} field in the receiver's IPC
buffer is set to the number of capabilities transferred up to failure.
No error message will be returned to the receiving thread in any of the
above cases.

\section{Asynchronous Endpoints}

Asynchronous endpoints (\obj{AsyncEP}s)
allow senders to perform IPC without blocking.
However, capabilities cannot be transferred over asynchronous endpoints.

Each asynchronous endpoint stores a single word of data. This word of
data may be written to using \apifunc{seL4\_Notify}{sel4_notify}, causing the
first message register of the sending thread to be bitwise or-ed with
the asynchronous endpoint's data word.

Note that \apifunc{seL4\_Notify}{sel4_notify} is not a proper system call
known by the kernel. Rather, it is a convenience
wrapper provided by the seL4 userland library which calls
\apifunc{seL4\_Send}{sel4_send} with a single parameter. It is
useful for notifying an asynchronous endpoint.

Additionally, the \apifunc{seL4\_Wait}{sel4_wait} system call may be used with an
asynchronous endpoint, allowing the calling thread to retrieve all set
bits from the asynchronous endpoint and clearing the endpoint in the
process. If no \apifunc{seL4\_Notify}{sel4_notify} system calls have taken place since the last
\apifunc{seL4\_Wait}{sel4_wait} call, the calling thread will block until the next
\apifunc{seL4\_Notify}{sel4_notify} takes place.

An asynchronous endpoint is the only object that may have
\apifunc{seL4\_Poll}{sel4_poll} used on it. If a message was received then the
returned badge will be a bitwise or of the badge(s) of the sender(s). Otherwise
the badge will be zero. Note that this means the receiver cannot detect
messages sent using unbadged capabilities, and thus the sender should use a
badged capability

\subsection{Asynchronous Endpoint Badges}

Like synchronous endpoints, asynchronous endpoint capabilities may also
be minted to create a new capability with a \emph{badge} attached to it (see \autoref{sec:ep-badges}).
The badge is a word of data that is associated with a particular
endpoint capability. When a message is sent to an endpoint using
a badged capability, the badge becomes part of the message received: in
particular, the badge is bitwise or-ed with the badges from
previous send system calls that have occurred since that last receive
on the endpoint.

Like synchronous endpoints, an asynchronous endpoint capability with
a zero badge is said to be \emph{unbadged}. Such a capability can be
badged with the \apifunc{seL4\_CNode\_Mutate}{cnode_mutate} or \apifunc{seL4\_CNode\_Mint}{cnode_mint} invocations on the
\obj{CNode} containing the capability. Asynchronous endpoint
capabilities with badges cannot be unbadged, rebadged or used to create
child capabilities with different badges.

\subsection{Binding Asynchronous Endpoints}
\label{sec:asyncep-binding}

\obj{AsyncEP}s and \obj{TCB}s can be bound together in a 1-to-1 relationship
through the \apifunc{seL4\_TCB\_BindAEP}{tcb_bindaep} invocation. When an
\obj{AsyncEP} is bound to a \obj{TCB}, then messages to that asynchronous
endpoint will be delivered if the thread is waiting on any synchronous
endpoint. The delivery will occur as if the thread had waited on the
asynchronous endpoint, with the label field of the \texttt{seL4\_MessageInfo}
structure set to \texttt{seL4\_Interrupt}.

Once an asynchronous endpoint has been bound, the only thread that may perform
\apifunc{seL4\_Wait}{sel4_wait} on the endpoint is the bound thread.
